\chapter{Level-7 Switch}\label{ch:method}

In this chapter, we present our Layer-7 switch using AF\_XDP technology and its application to a simulated maritime operational technology system.
In particular, we will highlight the differences between a traditional Layer-2/Layer-3 switch and our implementation, with a focus on its smart-filtered data flow.
The following sections describe in great detail the analysis steps leading to the final project architecture.

\section{From L2 to L7}
When we talk about a "layer" in switch terminology, we are referring to the Open System Interconnection (OSI) layer the switch operates on.
Today's implementations, applied to maritime OC systems, work on the the Data Link layer (L2) operating eventually on the Network Layer (L3).
The lower one being responsible for MAC-Address based forwarding of data frames.

Traditionally network device interfaces (NICs), each one with a unique MAC address, are connected to the switch via physical Ethernet ports or fiber optic ports depending on speed requirements.
When a NIC sends data to the switch its MAC address is inserted inside a very simple data structure, called the MAC address table, which contains basic routing information.
If the table contains the packet's destination address, the packet is routed to the associated port otherwise it is sent to all ports except the sender's, creating a packet "flood".

Requiring no routing algorithm, and not needing IP addresses to forward data, Layer 2 switches are very fast, and cost less than routers. However, broadcast traffic is not controlled leading to network congestion while handling high workloads. Lastly, these kind of switch cannot pass data between different VLANs precluding network segmentation.

Faced with the practical and performance limitations of lower-layer switches, there has been a move toward Layer 3 implementations, especially in industrial network topologies.
To ensure flexible application planning a host of Network Layer features are quickly became “must haves”.

Layer 3 packet routing is performed by routers that use IP addresses instead of MAC addresses, making this new paradigm ideal for local area networks. Switches that use this strategy can connect different VLANs, provide more security features, and apply Quality of Service (QoS) controls for maximum efficiency. 

Using an ARP table data structure which shows both MAC and IP addresses, switches will either forward the packet like a Layer 2 switch, or route it according to a routing protocols.

\leavevmode\newline
\begin{tabularx}{\textwidth}{|>{\hsize=0.5\hsize}X|>{\hsize=0.5\hsize}X|}
  \hline
  \multicolumn{2}{|c|}{\textbf{L2/L3 Features}} \\ \hline
  L2 & L3 \\ \hline
  Sends data “frames” to destination MAC address & Routes data “packets” based on MAC or IP address\\ \hline
  OSI Layer 2 (Data Link Layer) & OSI Layer 3 (Network Layer)\\ \hline
  Cannot connect different VLANs & Able to connect different VLANs\\ \hline
  One broadcast domain & Multiple broadcast domains \\ \hline
  Communicates with local network & Can connect to outside (multiple) networks\\ \hline
\end{tabularx}
\leavevmode\newline \\

