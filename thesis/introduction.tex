\chapter{Introduction}
\label{sec:introduction}

\section{Motivation and Context}

Operational Technology systems  are hardware and software solutions that monitor 
and control physical devices, processes, and infrastructure in industrial environments.\\
In this work we deal with the problem of simulating a naval one's internal
networking focusing on packet flow through the infrastructure's components.\\
The aforementioned are characterized by a state-of-the-art communication that is completely unrestricted, unfiltered and omnidirectional thereby facilitating uninhibited behavior within the network.
The motivation behind our effort is to a provide a drop-in solution to this main 
problem of marine electronic devices without introducing additional hardware nor new communication standards.\\
Such problem is originated by modern OT systems' nature :

\begin{itemize}
	\item The composition is characterized by the presence of legacy code, the optimization of which can be a laborious and time-consuming process that necessitates a considerable investment of resources
    \item The communication protocols used are entirely devoid of any encryption mechanism, thereby rendering information exchange remarkably vulnerable and unreliable
    \item The implementation of substantial modifications is not a straightforward process, as they are static assets that play a crucial role of fundamental systems that are indispensable to our daily lives
\end{itemize} 


\section{Goals}\label{sec:goals}

The Goal of this work is to implement a L7-Switch with the following features:
\begin{itemize}
    \item agnostic to the software and hardware of the running architecture with the ability to be integrated into existing infrastructures;
    \item capable of abstracting the naval enviroment components leveraging policy rules among them starting from simple configuration files;
	\item leverages the AFXDP technology for fast and efficient network packet processing;
    \item preserves normal communication flow for messages that do not conform to "National Marine Electronics Association (NMEA)";
    \item is able to process, filter and drop NMEA sentences considering policy rules previously defined;
    \item preserves legacy solutions' performances
\end{itemize}

\section{Content of the thesis}

We start by presenting the relevant background information (\autoref{ch:background}) of Software Defined Networking, Naval Protocols and Networking covering also the underlying technologies of the L7-Switch, EBPF and AFXDP.\\
We then proceed by analysing the Switch's architecture and functionalities (\autoref{ch:method}) discussing the normal packet flow and the filtered one.\\
Following we will review the implementation choices (\autoref{ch:implementation}) exploring the 
decisions behind the software's structure and pondering the resulting performances (\autoref{ch:evaluation}).
Lastly, we will present a selection of existing works and proposed solutions (\autoref{ch:related}).

