\chapter{Introduction}
\label{sec:introduction}

\section{Motivation and Context}
The motivation behind our effort is trying to provide a drop-in solution to Operational Techonlogy (OT) systems' main problem: protocols vulnerabilities in system-to-system communication.\\
OT systems  are hardware and software solutions that monitor 
and control physical devices, processes, and infrastructure in industrial environments.\\
The aforementioned are characterized by a internal communication structure that is completely unrestricted, unfiltered and omnidirectional, thereby facilitating uninhibited behavior within the network.
Finding a solution to this problem results in a particularly hard task because of OT systems' nature:
\begin{itemize}
	\item The composition is characterized by the presence of legacy code, the optimization of which can be a laborious and time-consuming process that necessitates a considerable investment of resources
    \item The communication protocols used are entirely devoid of any encryption mechanism, thereby rendering information exchange remarkably vulnerable and unreliable
    \item The implementation of substantial modifications is not a straightforward process, as they are static assets that play a crucial role of fundamental systems that are indispensable to our daily lives
\end{itemize} 

Leveraging these characteristics such solution should be \textbf{retrocompatible, transparent and should not introduce additional overhead}.

\section{Goals}\label{sec:goals}

The goal of this work is to implement a simulation of a OT Naval system which leverages a Layer7 software switch with the following features:
\begin{itemize}
    \item should be agnostic to the underlying software and hardware of the running architecture with the ability to be integrated seamlessly into existing infrastructures;
    \item should be capable of abstracting OT systems' components leveraging also policy rules that regulates communication among them;
    \item should preserve normal communication flow for messages that do not conform to "National Marine Electronics Association (NMEA)";
    \item should be able to process, filter and drop NMEA sentences considering policy rules previously defined;
    \item preserve or enhance legacy solutions' performances
\end{itemize}

The architecture we aim to build can be visualized as follows: 

\begin{figure}[H]
	\centering
    \includegraphics[scale=0.25]{thesis/images/architecture_overview.png}
	\caption{Project architecture visualization}
    \label{fig:project architecture visualisation}
\end{figure}

The final implementation will be, as displayed above, a tool whose purpose will be managing higher level ship component network traffic with the ability to forward, filter and drop information flowing inside of the vessel.  

\section{Content of the thesis}
The first topics presented are relevant background information (\autoref{ch:background}) of Naval Networks and Protocols, Software Defined Networking and the underlying technologies of the L7-Switch, EBPF and AFXDP.\\
Next the Switch's architecture and functionalities will be analyzed (\autoref{ch:method}) discussing the normal packet flow and the filtered one.\\
Following implementation choices will be reviewed (\autoref{ch:implementation}) exploring the 
decisions behind the project's structure and pondering the resulting performances (\autoref{ch:evaluation}).
Lastly a selection of existing works and proposed solutions will be presented considering fundamental aspects of each one (\autoref{ch:related}).


