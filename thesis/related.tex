\chapter{Related Works}\label{ch:related}

\section{OT Security overview}
In recent years, there has been a concerted effort to modernize and fortify critical maritime OT infrastructures. This initiative has given rise to the development of cybersecurity campaigns and directives, which have in turn identified opportunities to enhance the management of control systems and network operations. These are comprised of a set of rules and procedures that aim to raise awareness of potential cyber risks as well as to define the behaviors that are to be adopted, identifying the human factor as the most significant vulnerability, as stated in a serie of articles and online resources such as \cite{industry_approach}, \cite{securing_maritime}, \cite{maritime_decision_makers}, \cite{cyber_preparedness}.
While incentivizing solutions that do not require additional software nor hardware seems to be most popular choice, with great focus on personnel training, the need for a technical response to this menace has been highlighted in various other articles\cite{integrated_ship_cybersecurity}.
