\chapter{Results}
\label{ch:results}

% Here you will present the results of your work. Your skeptical reader is now asking themself
% ``is this working?''. You will show here to what extent it does. Be honest about the limitations:
% if your system doesn't work in some cases, you should say so (and explain where and why, if you
% can). Like the Method chapter, also this chapter may end up being split into
% multiple chapters.
%
% Keep in mind that this shouldn't be a mere dump of experimental results;
% you're rather \emph{teaching} your reader how and to what extent you managed
% to solve the problem you described in \Cref{sec:introduction}. If you have
% additional results that may be useful but are not necessary to understand the
% points you're making (e.g., you evaluated your system on multiple datasets
% and the results all tell the same story), the place for them is in an
% appendix.
%
% This chapter should have a lot of links to the methodology chapter(s),
% because you're evaluating the choices you made there. If you developed a
% system made of multiple parts, make sure that you test them separately and
% together, so that the reader can understand how important each part is.
%
% \begin{table} \centering \begin{tabular}{ccc} \toprule \textbf{Column 1} &
% \textbf{Column 2} & \textbf{Column 3} \\ \midrule 1 & 2 & 3 \\ 4 & 5 & 6 \\ 7
% & 8 & 9 \\ \bottomrule \end{tabular} \caption{A table using the \latex
% \texttt{booktabs} package. Note there are no vertical rules. In case you want
% to do more fancy stuff (e.g., merging cells), check also the
% \texttt{multirow} and \texttt{multicol} packages.} \label{tab:table}
% \end{table}
%
% This chapter is likely to be quite full of figures and tables. Try to make
% them as informative as possible (e.g., use multiple lines in the same plot if
% possible). Showing graphs effectively is a complex art; try to spend some
% time on it and ask for guidance from your advisor. Whenever you have a figure
% or a table, make sure that you refer to it in the text (after all, if it's
% not referred to in the text it means it has no part in the story you're
% telling, so it has no place in this Chapter). Always use the text (and the
% caption, if you can do it synthetically) to explain what you want the reader
% to understand from the figure.
%
% For Figures, generally use vectorial formats (e.g., PDF, SVG) where it makes
% sense if you can. They will result in higher-quality images that are in most
% cases also smaller, leading to shorter compiling times and a smaller
% resulting PDF.
%
% Tables look better without vertical rules: an example is in \Cref{tab:table}.
%
% \section{Discussion}
%
% At the end of this chapter, take a step back and summarize all the results so
% that the reader can understand the big picture. Sometimes, this becomes large
% and important enough to be worth a chapter on its own.

In this chapter we first present the accuracy of our simulator, comparing it
against the Matlab Antenna Toolbox and theoretical predictions. Then,
measure and compare its performances.

\section{Validation of the propagation model}

\subsection{Validation of received power}

In order to ensure our radar simulator propagates EM waves realistically, we
performed different tests against the ray tracing propagation model, based on
the SBR method (\autoref{sec:sbr}), implemented in the Matlab Antenna Toolbox
\cite{antennatoolbox}.

The Matlab implementation uses the deterministic approach described in
\autoref{sec:sbr}. It supports reflection but supports neither diffusion or
transmission, therefore we tested our reflective materials. First, we created a
perfectly specular material, which reflect the EM wave preserving all its
energy, independent of incident angle. While this idealized material is not
realistic, it is implemented in Matlab as well, offering a simple starting
point for our evaluation. The environment setup consists of a transmitting and
receiving antennas, placed at two meters from each others, and of a single
reflective plate with area of $1 \text m^2$ placed at varying distances from
the antennas. Both the transmitting and receiving antennas have the directivity
shown in \autoref{fig:testdirectivity}
with
the direction of peak directivity facing the plate. The scene is shown in
\autoref{fig:siteviewer}.
The common settings to Matlab and our implementation, used for the simulation,
are shown in
\autoref{table:testsettings}.
Furthermore, in Matlab we set the angle separation for the propagation model to
low, corresponding to 655362 rays \cite{raytracingmatlab}. For this reason, we
set the number of rays to 655362 in our simulation as well.

\begin{figure}
	\centering\includegraphics[scale=0.6]{images/directivity_used_in_test.pdf}
	\caption{Directivity in dB on the azimuth plane of the transmitting and
	receiving antennas used both in our implementation and in Matlab.}
	\label{fig:testdirectivity}
\end{figure}

\begin{figure}
	\centering\includegraphics[scale=0.8]{images/test_scene}
	\caption{Scene used for testing as shown in the Matlab site viewer for a
	plate distance of 50 m. The transmitting antenna is depicted in red, the
	receiving one in blue. Lines show the propagation paths.}
	\label{fig:siteviewer}
\end{figure}


\begin{table}
	\centering
	\begin{tabular}{ll}
		\toprule
			\textbf{Setting} & \textbf{Value} \\
			\hline
			Peak power & 1 MW \\
			Directivity & peak of 23.9 dB, pattern in \autoref{fig:testdirectivity} \\
			Frequency & 5 GHz \\
			Rays & 655362 \\
			Maximum number of reflections & 10 \\
			Plate material & perfectly reflective \\
			Plate shape & square  \\
			Plate area & $1~\text m^2$ \\
		\bottomrule
	\end{tabular}
	\caption{Settings of the simulation, used both in Matlab and in our implementation.}
	\label{table:testsettings}
\end{table}

\begin{table}
	\centering
	\begin{tabular}{ccc}
		\toprule
			\textbf{Plate distance} & \textbf{Matlab ($\dBm$)} & \textbf{Our implementation ($\dBm$)} \\
			\hline
			50 m  & 50.54 & $48.92 \pm 1 \cdot 10^{-4} $ \\
			100 m & 45.22 & $44.21 \pm 3 \cdot 10^{-4} $ \\
			150 m & 41.75 & $40.89 \pm 7 \cdot 10^{-5} $ \\
			200 m & 39.21 & $38.53 \pm 1 \cdot 10^{-4}$ \\
			250 m & 37.34 & $36.81 \pm 2 \cdot 10^{-4}$ \\
			300 m & 35.92 & $35.44 \pm 2 \cdot 10^{-4}$ \\
			350 m & 34.64 & $33.98 \pm 3 \cdot 10^{-4}$ \\
			400 m & {\color{red} 0.33} & $32.58 \pm 4 \cdot 10^{-4}$ \\
			450 m & {\color{red} 0.33} & $31.66 \pm 5 \cdot 10^{-4}$ \\
			500 m & {\color{red} 0.33} & $30.37 \pm 8 \cdot 10^{-4}$ \\
			1000 m & {\color{red} 0.33} & $24.37 \pm 3 \cdot 10^{-3}$ \\
			% 1500 m & 22.00 & $17.92 \pm 1 \cdot 10^{-5} $ \\
			% 1700 m & 21.52 & $15.00 \pm 9 \cdot 10^{-5} $ \\
			% 1800 m & 20.01 & {\color{red} 0.81} \\
		\bottomrule
	\end{tabular}
	\caption{Signal strength at the receiving antenna using Matlab and our
	implementation, in $\dBm$, for different distances of the reflective plate.
	Cells in red indicate that the simulator fails to account for the
	reflection of the plate. We can see that for Matlab this happens already for
	400 m. % while in our implementation the same happen for 1800 m.
	}
	\label{table:results}
\end{table}

\autoref{table:results} summarize the received power in $\dBm$ both in Matlab
and in our implementation. Since our implementation uses a randomized approach
for the generation of the rays, we computed the average power over 500
executions and the standard deviation. Results show a good degree of similarity
between the Matlab Antenna Toolbox and our implementation, despite the
different approaches.

\autoref{table:results} also shows that, when increasing the distance of the
plate, eventually, no ray will hit it. This is indicated in red in the table
and it is caused by the angular distance between rays: the further the rays
travel the bigger the gap among the rays. This makes possible for an object to
lay between rays without being hit, causing no reflection to be generated and
rendering the object undetectable, as depicted in \autoref{fig:angular-sep}.
The results show that this happen in Matlab already for a distance of 400 m,
while in our simulation, despite the number of rays being the same we can
correctly account for the reflection. The reason for this is that rays in our
simulation are not traced uniformly in all direction but focusses the rays in
the directions of high directivity, greatly increasing the ray density there -
and in turn the maximum distance an object can be detected at when placed in a
direction with high directivity - without requiring an increase in the number
of rays. 

\begin{figure}
	\centering

	\begin{subfigure}[t]{0.3\textwidth}
		\vfill
		\begin{tikzpicture}
			\newcommand*{\last}{29}
			\foreach \x in {0,...,\last} {
				\pgfmathsetmacro{\angle}{360.0 * \x / (\last + 1)}
				\draw[->] (0,0) -- (xyz polar cs:angle=\angle, radius=2.4);
			}
			\undef\last

			\fill[fill=red] (0, 1.1) circle (.15);
		\end{tikzpicture}
		\subcaption{The object is near and it is hit by some rays.}
	\end{subfigure}
	\hfill
	\begin{subfigure}[t]{0.3\textwidth}
		\vfill
		\centering\begin{tikzpicture}
			\newcommand*{\last}{29}
			\foreach \x in {0,...,\last} {
				\pgfmathsetmacro{\angle}{360.0 * \x / (\last + 1)}
				\draw[->] (0,0) -- (xyz polar cs:angle=\angle, radius=2.4);
			}
			\undef\last

			\fill (0, 2.2) circle (.15);
		\end{tikzpicture}
		\subcaption{The same object, placed further away from the source, is no
		longer hit by any rays.}
	\end{subfigure}
	\hfill
	\begin{subfigure}[t]{0.3\textwidth}
		\vfill
		\centering\begin{tikzpicture}
			\newcommand*{\last}{44}
			\foreach \x in {0,...,\last} {
				\pgfmathsetmacro{\angle}{360.0 * \x / (\last + 1)}
				\draw[->] (0,0) -- (xyz polar cs:angle=\angle, radius=2.4);
			}
			\undef\last

			\fill[fill=red] (0, 2.2) circle (.15);
		\end{tikzpicture}
		\subcaption{By decreasing the angular separation, the object placed far
		away can be hit again by rays.}
	\end{subfigure}
	\hfill

	\caption{The angular separation affects the ability to account for the
	reflection on far objects. If the angular separation is too big, the object
	may lay between rays and go undetected.}

	\label{fig:angular-sep}
\end{figure}



Next, we evaluated more realistic materials. In particular we created and
tested reflective materials defined by permittivity and conductivity, following
the recommendation of the Radiocommunication Sector of International
Telecommunication Union (ITU) \cite{itu-buildings}. Since we discovered with
the perfectly specular material that Matlab fails to account for the reflection
on the plate when placed further than 350m, we now limit ourselves to distances
from 50m to 350m. \autoref{table:concrete-material} presents the signal
strength for a plate using with permittivity and conductivity of concrete,
while \autoref{table:wood-material} uses the parameters of wood. In both cases
we note we get similar results to Matlab, with the greater discrepancy of
around $2 \dBm$. We can also notice that the standard deviation is higher than
the perfectly specular material. This is most likely caused by the fact that
when computing the reflected power with the Frensel equation, the result
depends on the angle of the incoming ray with the surface normal. Since we
generate the rays with random directions, the angle with the normal varies as
well, slightly affecting the stability of the result.


\begin{table}
	\centering\begin{tabular}{ccc}
		\toprule
		\textbf{Plate distance} & \textbf{Matlab ($\dBm$)} & \textbf{Our implementation ($\dBm$)} \\
		\hline
		50 m  & 42.44 & $41.05 \pm 4 \cdot 10^{-4} $ \\
		100 m & 37.20 & $36.08 \pm 3 \cdot 10^{-4}$ \\
		150 m & 33.66 & $32.75 \pm 1 \cdot 10^{-4}$ \\
		200 m & 30.97 & $30.39 \pm 2 \cdot 10^{-4}$ \\
		250 m & 29.17 & $28.65 \pm 3 \cdot 10^{-4}$ \\
		300 m & 27.98 & $27.31 \pm 5 \cdot 10^{-4}$ \\
		350 m & 26.80 & $25.85 \pm 7 \cdot 10^{-4}$ \\
		\bottomrule
	\end{tabular}
	\caption{Received power in $\dBm$ for a plate with the concrete material,
	placed at different distances.}
	\label{table:concrete-material}
\end{table}

\begin{table}
	\centering\begin{tabular}{ccc}
		\toprule
		\textbf{Plate distance} & \textbf{Matlab ($\dBm$)} & \textbf{Our implementation ($\dBm$)} \\
		\hline
		50 m  & 35.16 & $33.61 \pm 2 \cdot 10^{-3}$ \\
		100 m & 30.11 & $28.84 \pm 4 \cdot 10^{-4}$ \\
		150 m & 26.39 & $25.53 \pm 7 \cdot 10^{-4}$ \\
		200 m & 23.39 & $23.18 \pm 1 \cdot 10^{-3}$ \\
		250 m & 21.74 & $21.44 \pm 2 \cdot 10^{-3}$ \\
		300 m & 21.04 & $20.12 \pm 2 \cdot 10^{-3}$ \\
		350 m & 20.05 & $18.67 \pm 3 \cdot 10^{-3}$ \\
		\bottomrule
	\end{tabular}
	\caption{Received power in $\dBm$ for a plate with the wood material,
	placed at different distances.}
	\label{table:wood-material}
\end{table}


% Matlab concrete plate
% power
%    42.4402   37.2049   33.6603   30.9768   29.1674   27.9785   26.8030
%
% times
%     0.6636    0.0989    0.0831    0.0523    0.0379    0.0399    0.0341

% matlab wood plate
% power
%    35.1554   30.1109   26.3900   23.3903   21.7383   21.0421   20.0524
%
% times
%     0.6978    0.1002    0.0865    0.0562    0.0416    0.0453    0.0395
%
% Matlab metal plate
% power
%    50.5392   45.2135   41.7519   39.2112   37.3414   35.9203   34.6423
%
% times
%     0.0351    0.0411    0.0354    0.0352    0.0288    0.0326    0.0424


\subsection{Validation of Doppler shifts}

\begin{figure}\centering
	\includegraphics[width=0.6\textwidth]{images/doppler_away.pdf}
	\caption{Received power in Watts for frequencies around the emitted
	frequency of 5 GHz for a plate moving away from the radar with a speed of
	14 m/s. As expected for a target moving away from the radar, the received
	frequency is less than the transmitted one.}
	\label{fig:doppler-away}
\end{figure}


We now verify the presence of doppler shift due to the velocity of objects. 
Since the Matlab Antenna Toolbox does not support Doppler shifts we compare our simulator
against theoretical predictions.
For this test we placed a square reflective plate with side of 1 m at a distance of
100 m from the monostatic radar, moving away from it with a speed of 14 m/s
(50.4 km/h). \autoref{fig:doppler-away} shows the received power in Watts for
the frequencies around the transmitted one for the described setup. In particular,
from \autoref{eq:radardoppler}, by solving for v, we can get the speed along the line 
of sight:
\begin{equation}\label{eq:speed-from-shift}
	v = c \frac{f - f_r}{f + f_r}
\end{equation}
where $c$ is the speed of light, $f$ the transmitted frequency and $f_r$ the received 
frequency. In the described setup, the shifted frequency $f_r$ as yielded by
the simulator is 4999999533.01 Hz and the transmitted frequency $f$ is 5 GHz.
Using \autoref{eq:speed-from-shift} we get that the speed along the line of sight 
is 14.00 m/s, as expected.

Similarly, testing now an object moving towards the radar with a speed of $14$
m/s, we get an increase in frequency as depicted in
\autoref{fig:doppler-approaching}. In quantitative terms, the received
frequency is 5000000466.99 Hz, which, by using \autoref{eq:speed-from-shift},
yields the expected speed of $-14.00$ m/s. The negative sign indicates that the
target is approaching the radar.

\begin{figure}\centering
	\includegraphics[width=0.6\textwidth]{images/doppler_approaching.pdf}
	\caption{Received power in Watts for frequencies around the emitted
	frequency of 5 GHz for a plate moving towards the radar with a speed of
	14 m/s. As expected for a target approaching the radar, the received
	frequency is less than the transmitted one.}
	\label{fig:doppler-approaching}
\end{figure}



\section{Performance comparison}\label{sec:benchmark}

\subsection*{Benchmarking platforms}
% \glnote{Specifications, cost in the ballpark of 1000 euros, which means a low
% end commodity hardware, se paghi schizzi di piú, ad esempio una scheda cristo
% mi invidia fa N volte i core della mia quindi quello che vedete fa schifo ma
% potrebbe essere money issue rip balzello 2024-2024}

Performance comparisons presented in this section are run on the hardware and
software specifications given in \autoref{table:specs}. The RX 6650 XT video
card supports ray hardware accelerator (\autoref{sec:hw-rt}). However, being a
low-end device, it contains only 32 ray accelerators \cite{rx6650xt}. For
comparison, higher end devices from the same generation, such as the RX 6900 XT
\cite{rx6900xt}, contain 80 ray accelerators. Therefore we believe that the
performance figures obtained in the benchmarks in this section, while
significantly encouraging, would improve by a large amount with a
state-of-the-art GPU with more ray accelerators.

\begin{table}
	\centering
	\begin{tabular}{cc}
		\toprule
		\textbf{Setting} & \textbf{Value} \\
		\hline
		CPU & AMD Ryzen 5 5600X 6-Core Processor \\
		RAM & 32 GiB DDR4 \\
		GPU & AMD Radeon RX 6650 XT \\
		OS & Arch Linux, Linux 6.10.7 \\
		Vulkan Driver & RADV \\
		Vulkan Version & 1.3.279 \\
		Rust compiler & \texttt{rustc} 1.80.0 \\ 
		Matlab version & Matlab R2024a \\
		Frames in flight & 2 \\
		Transfer queue & On \\
		\bottomrule
	\end{tabular}
	\caption{Hardware and software specifications for the test environment.}
	\label{table:specs}
\end{table}


\subsection{Performance in a simple scene}

We present now how the performances of the Matlab Antenna Toolbox and our
implementation compare in the scene with a single plate presented in the
previous section. In particular we present the time required for the simulation
for the various distances of the plate. The hardware and software
specifications are detailed in \autoref{table:specs}, while the test scene is 
the same as \autoref{fig:siteviewer}.

The number of traced rays that we used with the Matlab Antenna Toolbox and with
our implementation for the test in \autoref{table:testsettings} is the same,
but performances are significantly different. In Matlab we measured the
performance for the ray tracing with in the following way:
\begin{minted}{matlab}
tic
rays = raytrace(tx,rx,pm);
toc
\end{minted}
where \texttt{tx} is the object representing the transmitting antenna,
\texttt{rx} is the object for the receiving antenna and \texttt{pm} is the
object representing the propagation model. In this way we measured only the time
required for the ray tracing and not the time required for the various initializations.
In our implementation we measured the time between subsequent steps of the
simulation for 500 steps to compute average and standard deviation. Results
are shown in \autoref{table:times}.
Results show that our simulation is from 35 to 50 times faster for the same
number of rays and in a identical scene.

\begin{table}
	\centering\begin{tabular}{cccc}
		\toprule
		\textbf{Plate distance} & \textbf{Matlab} & \textbf{Our implementation} & \textbf{Ratio} \\
		\hline
		50 m  & 39.5 ms & $0.8 \pm 0.3$ ms & 49.37 x \\
		100 m & 33.6 ms & $0.8 \pm 0.1$ ms & 42.00 x \\
		150 m & 39.0 ms & $0.8 \pm 0.1$ ms & 48.75 x \\
		200 m & 29.8 ms & $0.8 \pm 0.1$ ms & 37.25 x \\
		250 m & 28.3 ms & $0.8 \pm 0.1$ ms & 35.37 x \\
		300 m & 34.2 ms & $0.8 \pm 0.1$ ms & 42.75 x \\
		350 m & 34.4 ms & $0.8 \pm 0.1$ ms & 43.00 x \\
		\bottomrule
	\end{tabular}
	\caption{Runtime for the Matlab Antenna Toolbox and our implementation for
	the tracing of 655362 rays in the scene described in \autoref{table:testsettings}.
	We included the distances of the plates where the reflection is correctly
	accounted for.}
	\label{table:times}
\end{table}

\subsection{Performance in realistic scene}

The previously tested scene, containing only a plate, has only 4 vertices and
two triangles. Such a simple scene can hardly be considered representative of a
real world scenario. For this reason we benchmarked the Matlab Antenna Toolbox
and our implementation with a much more detailed scene, consisting of a 3D
model of the city of Shanghai \cite{shanghai-model}. This scene, containing
418k triangles and 338k vertices, is shown in \autoref{fig:shanghai-model}. For
this test, we placed the transmitting and receiving antennas in the middle of
the city, at a distance of 75 metres from each other and 3 metres from the
ground. Results in \autoref{table:shanghai-benchmark} shows how in this scene
our implementation takes on average 9.74 ms, while Matlab needs 843 ms. This
shows that in the more demanding scene Matlab becomes more than 86 times
slower than our implementation.

\begin{table}
	\centering\begin{tabular}{cccc}
		\toprule
		\textbf{Matlab} & \textbf{Our implementation} & \textbf{Ratio} \\
		\hline
		843.1 ms & $9.74 \pm 1.6 \cdot 10^{-5} $ ms & 86.56 x \\
		\bottomrule
	\end{tabular}
	\caption{Performance comparison in the Shanghai city model
	(\autoref{fig:shanghai-model}).
	Results shows that our implementation, in more dense scenes, further
	improves with respect to the Matlab Antenna Toolbox.}
	\label{table:shanghai-benchmark}
\end{table}

\begin{figure}
	\includegraphics[scale=.5]{images/shangai.png}
	\caption{Model of the city of Shanghai by Micheal Zhang
	\cite{shanghai-model} and provided under the Creative Commons Attribution
	4.0 International license
	\cite{cc-licence}. The original model included text with names of the most
	prominent buildings and has been removed for the use in this work. The
	scene consists of roughly 418k triangles and 338.6k vertices.}
	\label{fig:shanghai-model}
\end{figure}

\subsection{Comparison between hardware accelerated ray tracing and software ray tracing}

The AMD RX 6650XT GPU, used for performance tests in this work, provides ray
accelerators that speed up the computation of the intersection between rays and
triangles. We now determine how much the hardware ray tracing improves
performances with respect to traditional software, but still running on GPU,
ray tracing. A possible way to achieve this would be to compare the execution
times between a GPU with hardware ray tracing and one without. This method
would produce results hard to compare since two different GPU would differ in
other specifications apart from the presence of ray accelerators. Therefore we
decided to leverage the \verb!RADV_PERFTEST=emulate_rt! \cite{mesa-envvars}
environment variable, provided by the RADV driver, which runs the Vulkan ray
tracing pipeline (\autoref{subsec:raytracing}) completely in software. We
believe this to produce comparable results as the program can be run on the
same system without any modifications, with just the difference of not using
the ray accelerators. Results are presented in \autoref{table:emulate-rt}. The
performance difference is significant: running the ray tracing pipeline without
the help of the ray accelerators increases the simulation time by a factor of
6.28. This shows how it is important to make full use of the capabilities
provided by the hardware.

\begin{table}
	\centering\begin{tabular}{ccc}
		\toprule
		\textbf{Software ray tracing} & \textbf{Hardware ray tracing} & \textbf{Ratio} \\
		\hline
		$61.2 \pm 5.2$ ms & $9.74 \pm 1.6 \cdot 10^{-5}$ ms & 6.28 x \\
		\bottomrule
	\end{tabular}

	\caption{Performance comparison between running our simulator with
	\texttt{RADV\_PERFTEST=emulate\_rt} \cite{mesa-envvars}, to force the execution
	of the Vulkan ray tracing pipeline to run in software, and without, to use
	the hardware ray accelerators.}
	\label{table:emulate-rt}

\end{table}

\subsection{Effects of frames in flight and transfer queue on performances}

We now test how the number of frames in flight and the use of a dedicated
transfer queue (\autoref{sec:architecture}) affects the performances of the
radar simulator. The RADV driver we use to run our simulation at the moment of
this writing has no stable support for a transfer queue, but experimental
support exists and can be enable with the environment variable
\verb!RADV_PERFTEST=transfer_queue! \cite{mesa-envvars}. We first test the
simulation in scene containing a single plate, as previously described in
\autoref{table:testsettings} and distance of the plate of 150 m. Timings for
500 simulation steps have been measured and averaged when using from 1 to 3
frames in flight and with and without the dedicated transfer queue. Results,
reported in \autoref{table:plate-frame-flight}, show that using 2 frames in
flight improves performances with respect to a single frame in flight. However,
increasing the number of frames in flight to 3 provides no benefits. The reason
for this is most likely that two frames in flight suffices at keeping the GPU
utilized. Also, the use of the transfer queue further improves performances,
especially when combined with two frames in flight.

\definecolor{best}{rgb}{0.8,1,.8}
\begin{table}\centering
	\begin{tabular}{cccc}
		\toprule
		\textbf{Frames in flight} & \textbf{Transfer queue} & \textbf{Time} & \textbf{Ratio with best} \\
		\hline
		1 & Off & 1.24 ms & 1.63\\
		1 & On & 1.11 ms & 1.46\\

		2 & Off & 0.97 ms & 1.28 \\
		\cellcolor{best}2 & \cellcolor{best}On & \cellcolor{best}0.76 ms & \cellcolor{best}1.00 \\

		3 & Off & 0.97 ms & 1.28 \\
		3 & On & 0.76 ms & 1.00 \\
		\bottomrule
	\end{tabular}
	\caption{Time between frames (averaged from 500 frames) in a scene
	containing a single reflective plate, placed at 150 m from the receiving
	and transmitting antennas for different values of frames in flight and with
	or without using the transfer queue. Results show that numbers of frames in
	flight greater than 2 provides no benefits. Results also show that using
	more than one frame in flight with the transfer queue improves
	performances.}
	\label{table:plate-frame-flight}
\end{table}

We now test the more demanding scene containing the model of the city of
Shanghai (\autoref{fig:shanghai-model}). Results are shown in
\autoref{table:shanghai-frame-flight}. In this setup the best configuration is
again two frames in flight with the transfer queue. However, the performance
penalty from using other configuration is less severe than the previous case.
The reason for this, resides in the fact that, increasing significantly the
number of triangles in the scene, makes the simulation more GPU compute
limited, therefore the CPU time and transfer time are a less significant part
of the total.

\begin{table}\centering
	\begin{tabular}{cccc}
		\toprule
		\textbf{Frames in flight} & \textbf{Transfer queue} & \textbf{Time} & \textbf{Ratio with best} \\
		\hline
		1 & Off & 10.69 ms & 1.10 \\
		1 & On & 10.46 ms & 1.07 \\

		2 & Off & 10.57 ms & 1.08 \\
		\cellcolor{best}2 & \cellcolor{best}On & \cellcolor{best}9.74 ms & \cellcolor{best}1.00 \\

		3 & Off & 10.40 ms & 1.07 \\
		3 & On & 9.86 ms & 1.01 \\
		\bottomrule
	\end{tabular}
	\caption{Time between frames (averaged from 500 frames) in the scene
	containing the model of the city of Shanghai (\autoref{fig:shanghai-model})
	for different values of frames in flight and with or without using the
	transfer queue. Once again, results show that numbers of frames in flight
	greater than 2 provides no benefits. The best configuration is again 2
	frames in flight with the transfer queue.}
	\label{table:shanghai-frame-flight}
\end{table}

% matlab 150
% frames_in_flight=1 transfer_queue=1 time=0.0011488975647217287 +- 0.00021746955794604554
% frames_in_flight=1 transfer_queue=0 time=0.0012485441081748457 +- 0.0002745109556709025

% frames_in_flight=2 transfer_queue=1 time=0.0007636719094010775 +- 0.00014815791196936512
% frames_in_flight=2 transfer_queue=0 time=0.0009748109118016306 +- 0.0009748109118016306

% frames_in_flight=3 transfer_queue=1 time=0.0007646423064635129 +- 0.00021343140412292738
% frames_in_flight=3 transfer_queue=0 time=0.0009737683680348979 +- 0.00020556079588044402

% benchmark
% frames_in_flight=1 transfer_queue=1 time=0.010457601719246598 +- 1.9560468327610026e-05
% frames_in_flight=1 transfer_queue=0 time=0.010698397317248021 +- 0.00020252860224205865

% frames_in_flight=2 transfer_queue=1 time=0.009742900699317335 +- 1.6385037170188795e-05
% frames_in_flight=2 transfer_queue=0 time=0.01057120649991389 +- 0.00012082649257093778

% frames_in_flight=3 transfer_queue=1 time=0.009859281933618213 +- 3.247024996480421e-05
% frames_in_flight=3 transfer_queue=0 time=0.010406997256384106 +- 8.773811357408177e-06



\section{Qualitative results}

In this section we verify that various phenomena happen and are recognizable in
our simulation. We start by creating a range-azimuth plot of the returns
obtained from the simulation to verify that the object placed in the scene
appear in the right place. We first test a scene containing three spheres, each
using the diffuse material, placed at $(200\unit{m}, 0\unit m, 0\unit m)$, $(0
\unit m, 150\unit m, 0\unit m)$ and $(-100\unit m, 0\unit m, 0\unit m)$, each with a
radius of 5 metres. The radar is monostatic and placed at the origin $(0\unit
m, 0 \unit m, 0\unit m)$, with the antenna pattern shown
\autoref{fig:testdirectivity}. The received power in range and azimuth is shown
\autoref{fig:test_range_azimuth}. Three strong echoes are readily recognizable,
one for each of the spheres used in the scene. We can verify that the positions
of the echoes is correct as well, since they appear at a range 195 metres and
azimuth 0\textdegree, at a range of 145 metres and azimuth 90\textdegree~and at
a range of 95 metres and azimuth of 180\textdegree. Since the reflections
happen on the surface of the spheres, they appear closer than the centers by a
distance equal to the radius. Furthermore, we can also notice that the closer
the object the stronger the received power.

\begin{figure}
	\centering\includegraphics[scale=0.8]{images/test_range_azimuth}
	\caption{Range-azimuth plot of the received power, in $dBm$, for a scene
	containing three sphere of diffuse material placed at $(200\unit{m}, 0\unit
	m, 0\unit m)$, $(0 \unit m, 150\unit m, 0\unit m)$ and $(-100\unit m,
	0\unit m, 0\unit m)$, each with radius of 5\unit m.}
	\label{fig:test_range_azimuth}
\end{figure}

Next we evaluated the presence of false echoes. False echoes are caused by
multiple reflection in the environment, which causes the EM waves to travel a
distance longer than the actual distance between the radar and the objects. For
this test, we use a scene containing two plates, with the diffuse material,
placed at $(-75\unit m, 0\unit m, 0\unit m)$ and $(75\unit m, 0\unit m, 0\unit
m)$. The plates are square with the side of 10 m, parallel to each other and
perpendicular to the unit vector $\hat{\mathbf{x}}$. The radar is at the origin
and therefore sits between the two plates. \autoref{fig:false-echoes} shows the
received power in range and azimuth. The two nearest echoes correspond to the
actual plates and are indeed at the correct range of 75\unit m. Other two
echoes are also visible, at more than 200\unit m of range, behind both plates.
Those echoes do not correspond to any object at that position but are instead
caused by multiple reflection on the plates.

\begin{figure}
	\centering\includegraphics[scale=0.8]{images/multipath.pdf}
	\caption{Range-azimuth plot of the received power, in $dBm$, for two
	parallel plates, with the radar between them. Two strong echoes are visible
	in correspondence of the plates at 75 and -75 metres on the x axis. Two
	false echoes, caused by the multiple reflection between the plates, are
	also visible. }
	\label{fig:false-echoes}
\end{figure}


