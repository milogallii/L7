\chapter{Background}
\label{ch:background}

In this chapter, we delineate the principles of contemporary naval communications, emphasizing the characteristics and deficiencies of prevailing standards (NMEA).
The subsequent discussion will address the necessity for "software-defined networking" and the innovative technologies underpinning network administration, EBPF and AFXDP.


\section{Naval Networks}\label{sec:naval_networking_and_nmea}

A naval network, otherwise referred to as an \textbf{integrated shipboard communication system} is a sophisticated and meticulously organized system designed to facilitate communication, data exchange, and operational coordination whithin subsystems and devices on the vessel. 
Given the numerous elements that contribute to the ship's apparatus, such as navigation systems, sensors, monitoring systems, and combat-management systems, our focus will be directed exclusively towards the communication limb that directly interests our final goal.

\subsection{Communication systems}
Communication systems facilitate seamless communication of the vessel both internally and externally ensuring the continuous exchange of data.
The field of external communication can be subdivided into three primary categories :

\begin{itemize}
    \item\textbf{Command and Control (C2)},for long-range capabilities such as communication with shore-based command centers, other ships and aircrafts
    \item\textbf{Tactical Coordination}, for short-to-medium range communication which are critical for coordination and emergency communications
    \item\textbf{Emergency Communication},for real time data exchange among various components allowing coordinated operations among ships, aircrafts and ground forces
\end{itemize}

External side communication's capstone can be identified in the usage of the Satellite Communication (SATCOM) protocol that enables beyond-line-of-sight (BLOS) data exchange providing global, high-bandwidth, and secure connectivity. However, it is part of a broader ecosystem that includes radio communication, tactical data links, and underwater communication systems.\\
Internal communication, on the other hand, although it comprises crew coordination systems the main body consists of devices that provide a unified picture among navigation, propulsion, combat, and management environments. Vessels are designed to establish an informational circulatory system within them that consists of a physical apparatus composed of various components such as switches, routers, gateways, and servers. The flow of information within it falls into five distinct categories: 

\begin{itemize}
    \item\textbf{Navigation and Situational Awareness}, data transmitted over the network and integrated into the Combat Management System, providing a real-time operational picture
    \item\textbf{Threat Detection and Response}, data signaling potential threats coming from the sensoring systems
    \item\textbf{Engine Monitoring and Control}, data allowing the crew to monitor and optimize the ship's propulsion and power systems
    \item\textbf{Crew Communication}, intercom system's data to coordinate crew actions during missions
    \item\textbf{Cybersecurity Monitoring}, cybersecurity systems data that ensure integrity and safety of the ship
\end{itemize}

\subsection{NMEA}

Given the wide range of the aforementioned messages in the early years of 1980 the \textbf{National Marine Electronics Association (NMEA)} introduced the first kind of protocol to standardize communication for marine electronics.
Since its inception, the protocol has evolved significantly to accommodate advancements in technology and the expanding needs of different industries.

\subsubsection*{NMEA 0183}
The protocol underwent a series of modifications, beginning with the 0180 and 0182 versions. However, these initial implementations were swiftly discarded in favor of the subsequent iteration, the 0183. This new version's adoption as the primary standard for serial communications can be attributed to its simplicity and comprehensive functionality, providing the vessel's components a 
complete view over others' status.

This standard utilizes a rudimentary ASCII serial communications protocol, which delineates the method by which data are transmitted in a \textbf{sentence} from a single \textbf{talker} to multiple \textbf{listeners} concurrently. The protocol enables a talker to engage in a unidirectional conversation with an almost limitless number of listeners by employing intermediate expanders, and it facilitates communication between multiple sensors and a singular port through the use of multiplexers.  

Although the protocol stipulates the use of RS-422 for electrical transport, a de facto standard has emerged in which sentences are encapsulated in \textbf{UDP datagrams} and transmitted over IP networks.

\paragraph{Sentence structure}
\leavevmode\newline\\
In order to conform to the protocol all messages must be printable ASCII characters between \textit{0x20} (space) to \textit{0x7e} (tilde) including a set of special characters that define the syntax of sentences

\begin{tabularx}{\textwidth}{|>{\hsize=0.5\hsize}X|>{\hsize=0.5\hsize}X|>{\hsize=1\hsize}X|}
  \hline
  \multicolumn{3}{|c|}{\textbf{Syntax characters}} \\ \hline
  ASCII & HEX & USE \\ \hline
  \textless CR\textgreater  &  0x0d   & Carriage return        \\ \hline
  \textless LF\textgreater   & 0x0a    & Line feed, end delimiter       \\ \hline
  !    & 0x21   &   Start of encapsulation sentence delimiter   \\ \hline
  \textdollar     & 0x24   & Start delimiter      \\ \hline
  *    & 0x2a    & Checksum delimiter     \\ \hline
  ,    & 0x2c   & Field delimiter       \\ \hline
  \textbackslash     & 0x5c   & TAG block delimiter        \\ \hline
  \textasciicircum    & 0x5e     & Code delimiter for HEX representation of ISO/IEC 8859-1 (ASCII) characters     \\ \hline
  \textasciitilde     & 0x7e    & Reserved       \\ \hline
\end{tabularx}

\subsubsection{NMEA 2000}
\subsubsection{NMEA OneNet}




%\begin{figure}
	%\centering\includegraphics[scale=0.7]{images/directivity}
	%\caption{Directivity of a parabolic reflector antenna for $\theta = \frac{\pi}{2}$.
	%The angle is $\phi$, while the distance is the directivity expressed in
	%decibels.}\label{fig:directivity}
%\end{figure}

