\chapter{Background}
\label{ch:background}

In this chapter, we delineate the principles of contemporary naval communications, emphasizing the characteristics and deficiencies of prevailing standards (NMEA).
The subsequent discussion will address the necessity for "software-defined networking" and the innovative technologies underpinning network administration, EBPF and AFXDP.


\section{Naval Networks}\label{sec:naval_networking_and_nmea}

A naval network, otherwise referred to as an \textbf{integrated shipboard communication system} is a sophisticated and meticulously organized system designed to facilitate communication, data exchange, and operational coordination whithin subsystems and devices on the vessel. 
Given the numerous elements that contribute to the ship's apparatus, such as navigation systems, sensors, monitoring systems, and combat-management systems, our focus will be directed exclusively towards the communication limb that directly interests our final goal.

\subsection{Communication systems}
Communication systems facilitate seamless communication of the vessel both internally and externally ensuring the continuous exchange of data.
The field of external communication can be subdivided into three primary categories :

\begin{itemize}
    \item\textbf{Command and Control (C2)},for long-range capabilities such as communication with shore-based command centers, other ships and aircrafts
    \item\textbf{Tactical Coordination}, for short-to-medium range communication which are critical for coordination and emergency communications
    \item\textbf{Emergency Communication},for real time data exchange among various components allowing coordinated operations among ships, aircrafts and ground forces
\end{itemize}

External side communication's capstone can be identified in the usage of the Satellite Communication (SATCOM) protocol that enables beyond-line-of-sight (BLOS) data exchange providing global, high-bandwidth, and secure connectivity. However, it is part of a broader ecosystem that includes radio communication, tactical data links, and underwater communication systems.\\
Internal communication, on the other hand, although it comprises crew coordination systems the main body consists of devices that provide a unified picture among navigation, propulsion, combat, and management environments. Vessels are designed to establish an informational circulatory system within them that consists of a physical apparatus composed of various components such as switches, routers, gateways, and servers. The flow of information within it falls into five distinct categories: 

\begin{itemize}
    \item\textbf{Navigation and Situational Awareness}, data transmitted over the network and integrated into the Combat Management System, providing a real-time operational picture
    \item\textbf{Threat Detection and Response}, data signaling potential threats coming from the sensoring systems
    \item\textbf{Engine Monitoring and Control}, data allowing the crew to monitor and optimize the ship's propulsion and power systems
    \item\textbf{Crew Communication}, intercom system's data to coordinate crew actions during missions
    \item\textbf{Cybersecurity Monitoring}, cybersecurity systems data that ensure integrity and safety of the ship
\end{itemize}

\subsection{NMEA}

Given the wide range of values that messages in maritime environments can assume, with different meanings corresponding to each value, in the early years of 1980 \textbf{National Marine Electronics Association (NMEA)} introduced the first kind of protocol to standardize communication protocols for marine electronics.




%\begin{figure}
	%\centering\includegraphics[scale=0.7]{images/directivity}
	%\caption{Directivity of a parabolic reflector antenna for $\theta = \frac{\pi}{2}$.
	%The angle is $\phi$, while the distance is the directivity expressed in
	%decibels.}\label{fig:directivity}
%\end{figure}

