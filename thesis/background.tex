\chapter{Background}
\label{ch:background}

In this chapter, we delineate the principles of contemporary naval communications, emphasizing the characteristics and deficiencies of prevailing standards (NMEA).
The subsequent discussion will address the necessity for "software-defined networking" and the innovative technologies underpinning network administration, EBPF and AFXDP.


\section{Naval Networks}\label{sec:naval_networking_and_nmea}

A naval network, otherwise referred to as an "integrated shipboard communication system," is a sophisticated and meticulously organized system designed to facilitate communication, data exchange, and operational coordination among various subsystems and devices on the vessel. Given the numerous elements that contribute to the ship's apparatus, such as navigation systems, sensors, monitoring systems, and combat-management systems, our focus will be directed exclusively towards the communication aspect that directly interests our software's development.

\subsection{Communication systems}
Communication systems facilitate seamless communication both within the vessel and with external entities, ensuring the continuous exchange of data.

The field of external communication can be subdivided into three primary categories :
\begin{itemize}
    \item\textbf{Command and Control (C2)}, for long-range capabilities such as communication with shore-based command centers, other ships and aircrafts
    \item\textbf{Tactical Coordination}, for short-to-medium range communication and  critical for coordination and emergency communications
    \item\textbf{Emergency Communication}, which enable real time data exchange among ships, aircrafts and ground forces allowing coordinated operations
\end{itemize}


The internal communication body instead comprises crew coordination systems and primarily consists of devices that provide a unified picture among navigation, propulsion, combat, and management environments.

%\begin{figure}
	%\centering\includegraphics[scale=0.7]{images/directivity}
	%\caption{Directivity of a parabolic reflector antenna for $\theta = \frac{\pi}{2}$.
	%The angle is $\phi$, while the distance is the directivity expressed in
	%decibels.}\label{fig:directivity}
%\end{figure}
