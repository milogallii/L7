\chapter{Background}
\label{ch:background}

% This chapter provides the necessary background information to understand the content of your thesis.
% The goal is to make your thesis \emph{self-contained}: the reader, who you can assume will be
% a fellow Computer Science student, should be able to understand your work without having to refer
% to external sources. Of course, you should reference the sources you used to write this chapter
% and references to documents that the reader can use to know more about those topics.

% Don't include information that is not necessary to understand your work. If something is
% complementary to your work, but not necessary to understand it, you can include it in the
% Related Work chapter (\Cref{sec:related}).

% The Background chapter contains information about work that \emph{you didn't do}. The work you
% did should be decscribed in the Methodology chapter (\Cref{sec:method}) instead.

% The amount of information you will need to present in this Chapter can vary a lot between theses.
% In some cases, you will have no or very little need for a Background chapter; in that case, it
% may make sense to move this information to a section in the Related Work chapter.

In this chapter we present the fundamental principles of radar systems and an
overview of the physical properties of electromagnetic waves needed to
understand this work. Then, we present an overview of the Vulkan API, used 
in this work, with a particular focus on ray tracing capabilities, comparing it 
with other GPU programming APIs.

\section{Primary Radars}\label{sec:radarbackground}

A radar is a system that determines distance, direction and velocity of objects
(\emph{targets}) relative to its position. To achieve this, a radar transmits
electromagnetic waves toward a region, measuring the reflection (\emph{echoes})
returning from the objects \cite{richards2010principles}. A radar that receives
the reflection is also referred as primary radar, as opposed to the secondary
radar which expects a signal sent by a transponder. Because primary radars do
not require the target to be equipped with any specific device, they can be
used for a wider range of applications, from the creation of safety system in
the automotive industry \cite{automotive}, weather measures \cite{weather},
quantification of bird migrations \cite{schmaljohann2008quantification},
air-traffic control \cite{radarhandbook} and navigation systems.

More specifically, a radar can be described as made of a transmitter, an
antenna and a receiver \cite{richards2010principles}. The transmitter generates
the EM waves, which are then propagated by the antenna in the environment,
typically the atmosphere. This transmitted signal travels to the target which
radiates it again in the environment. Part of this reradiated signal reaches
again the antenna, which picks it up. The received signal then goes to the
receiver subsystem which processes it. The receiving antenna can be the same as
the transmitting antenna, in this case the radar is called \emph{monostatic},
or can be a different one, such a radar is called \emph{bistatic}.


\subsection{Information available from a radar}

\subsubsection{Range}

A radar determines the distance, or range, to a target by measuring the time
that the transmitted signal takes to travel from the radar to the target and
back. In particular, assuming the signal travels directly to the target and
back, the range $R$ of target can be computed from the time $\Delta T$ between
the emission of the signal and its return with \cite{richards2010principles}:
\begin{equation}
	R = \frac{c\Delta T}{2}
\end{equation}
where $c$ is the speed of light. In order to determine $\Delta T$, the radar
must establish when the waveform was transmitted. Commonly used ways to
accomplish this are to either modulate the signal in amplitude, by sending 
short pulses, or modulate the frequency \cite{radarhandbook}:
in the former case, the radar is called a \emph{pulsed} radar while in the
latter it is referred as \emph{continuous wave frequency modulated} radar.

In particular, a pulsed radar transmits EM waves for a short time $\tau$, which
can be as little as a nanosecond or as long as a millisecond. During the
transmission the receiver is isolated in order to protect it from the high
energy EM waves being transmitted, thus no signal can be received during the
transmission of the pulse. In between the transmission of two pulses, the receiver
can safely receive the echoes of the transmitted pulse. The sum of the time
spent receiving and the pulse width $\tau$ is the \emph{pulse repetition time (PRI)}.
Since a received echo is assumed to be a reflection of the last transmitted pulse,
it is possible, for the echo of a sufficiently distant target, to be received
after the transmission of the next pulse, causing an erroneous range
measurement. Therefore, for a pulsed radar the maximum unambiguous range,
$R_{max}$, can be defined as the longest range a pulse can travel between two
consecutive pulses and it depends on the PRI \cite[p.~22]{richards2010principles}:
\begin{equation}
	R_{max} = \frac{c \text{PRI}}{2}
\end{equation}
where $c$ is the speed of light. Therefore, increasing the PRI  allows for a
greater maximum unambiguous range. On the other hand, increasing the pulse
width allows for a transmission of more energy, rendering the detection of a
target easier in the presence of noise.

A continuous-wave radar transmits EM waves all the time without interruptions
\cite[p.~20]{richards2010principles}. For this reason, in order to determine
the round trip time of the transmitted EM waves, used for the range
calculation, the frequency is changed over time, putting a time mark on it.
Unmodulated continuous-wave radar can be used wherever it is only required to
measure the speed of the target and no range is needed.



\subsubsection{Angular direction}

% explain in layman terms illumination before dropping the term

% same as the concept of lobes, maybe refer the reader to the next section about antennas

A radar uses an antenna with a high directivity (\autoref{gain}), in such a way
that, in given moment, only a specific region is illuminated by the transmitted
EM waves. Thus, the direction of the target can be determined simply by finding
the direction the main beam of the antenna is facing. It is assumed that
targets outside of the main lobe return no echo (or a negligible one). By
denoting the direction by the elevation angle $\theta$ and the azimuth angle
$\phi$, together with the range $R$, we can express naturally the position of a
target in term of a spherical coordinate system with the origin at the radar.
Therefore, the detected target is at coordinates $(R, \theta, \phi)$.


\subsubsection{Range rate}

As already mentioned, radar systems can measure the radial velocity, or range
rate, of a target. This is possible thanks to the Doppler effect: when an
observer moves with respect to the source of a wave a shift in frequency is
measured. In the case of a target moving with a speed $v$ towards the radar,
the received frequency $f_r$ is \cite[p.~274]{richards2010principles}:
\begin{equation}\label{eq:radardoppler}
	f_r = \left(\frac{c + v}{c - v}\right)f
\end{equation}
where $c$ is the speed of light, and $f$ is the original frequency.
% In the radar setting with a target with a non-zero radial velocity, the doppler effect
% happens twice: first when the target is hit by the incoming wave and again when
% the echo is reradiated. From \autoref{eq:radardoppler} we can
% therefore obtain the $f_d$ the radar observes:
% \begin{equation}
% 	f_d = 2 \frac{v_r}{\lambda}
% \end{equation}

In qualitative terms, \autoref{eq:radardoppler} shows a target moving towards the radar
increases the received frequency, while a target receding decreases it.


\subsection{Basic antenna quantities}

\subsubsection{Radiation efficiency} In an antenna, part of power fed to it is
absorbed before being radiated. Therefore we can define the radiation
efficiency $\eta$ of an antenna as the ratio of the power radiated
$\powradiated$ over the power $\powtransmitter$ accepted by the antenna from
the connected transmitter:
% accepted ? radiated by the transmitting antenna

\begin{equation}\label{eq:efficiency}
	\eta = \frac{\powradiated}{\powtransmitter}
\end{equation}

\subsubsection{Directivity and Gain}\label{gain}

An antenna usually does not radiate isotropically, i.e.~uniformly in all
directions, but radiates most of the power in a certain direction. For this
reason it is useful to introduce the radiation intensity
$U(\theta, \phi)$, which represents the power per unit of solid angle and it is
expressed in Watts per steradian. $\theta$ and $\phi$ represent a direction in
spherical coordinates, with the antenna at the origin. By definition, for an
isotropic antenna the radiation intensity is constant over all directions,
while in general it is not. Furthermore, the total power
radiated by the antenna can then be expressed as a function of the radiation
intensity \cite[p.~5]{antennahandbook3}:
\begin{equation}\label{eq:radiationintensity}
	\powradiated = \int_0^{2\pi}\int_0^\pi U(\theta, \phi) \sin \theta d\theta d\phi 
\end{equation}

A useful quantity related to the radiation intensity is the antenna directivity
is defined as a unit-less quantity representing the ratio of the radiation
intensity of the antenna and the radiation intensity of an isotropic antenna
radiating the same total power:
\begin{equation}\label{eq:directivity}
	D(\theta, \phi) = \frac{U(\theta, \phi)}{\powradiated / 4\pi}
\end{equation}
The directivity characterizes how much and in which direction the antenna
is able to radiate the power. An example of directivity is plotted
in \autoref{fig:directivity}. A quantity related to the directivity is the
antenna gain, defined as \cite[p.~5]{antennahandbook3}:
\begin{equation}
	G(\theta, \phi) = \eta D(\theta, \phi)
\end{equation}
and therefore also accounts for the radiation efficiency of the antenna. By
using
\autoref{eq:efficiency} we can rewrite the gain as the radiation
intensity over the radiation intensity of the isotropic antenna that radiates
all the input power ($\eta = 1$):
\begin{equation}
	G(\theta, \phi) = \eta D(\theta, \phi) = \eta \frac{U(\theta, \phi)}{\powradiated / 4\pi} =
	\frac{U(\theta, \phi)}{\powtransmitter / 4\pi}
\end{equation}


\begin{figure}
	\centering\includegraphics[scale=0.7]{images/directivity}
	\caption{Directivity of a parabolic reflector antenna for $\theta = \frac{\pi}{2}$.
	The angle is $\phi$, while the distance is the directivity expressed in
	decibels.}\label{fig:directivity}
\end{figure}

\subsubsection{Effective area}

The ability of an antenna to receive power is represented by its effective area
$A_e(\theta, \phi)$. It is usually expressed 
as a function of the gain as follows \cite[p.~6]{antennahandbook3}:
\begin{equation}\label{eq:effectivearea}
	A_e(\theta, \phi) = \frac{\lambda^2}{4\pi}G(\theta, \phi)
\end{equation}
where $\lambda$ is the wavelength of the EM wave (see \autoref{sec:emwaves}).

\subsection{The radar equation}\label{subsec:radarequation}

When an echo reaches the radar, its power determines whether the target is
detected: a signal too weak cannot be distinguished from the environmental noise
constantly reaching the radar. The power of the received EM waves for a target
at range $R$ is described by the radar equation.
First we compute the power density
$Q_i$ at the target:
\begin{equation}
	Q_i = \frac{\powtransmitter G_t}{4\pi R^2}
\end{equation}
where $G_t$ is the gain in the direction of the target.
The power reflected by the target is given by:
\begin{equation}
	P_{refl} = Q_i \sigma = \frac{\powtransmitter G_t \sigma }{4\pi R^2}
\end{equation}
where $\sigma$ is the radar cross section of the target. The radar cross
section is a measure of how detectable a target is and it depends on the
material, shape, size and both the angles of the incident and reflected waves.
With power reflected by the target, we can now compute the power density $Q_r$
at the radar:
\begin{equation}
	Q_r = \frac{P_{refl}}{4\pi R^2} = \frac{\powtransmitter G_t \sigma }{(4\pi)^2 R^4} 
\end{equation}

To finally compute the power received by the radar, we simply multiply the 
power density with the effective area:
\begin{equation}
	P_r = Q_r A_e = \frac{\powtransmitter G_t \sigma A_e}{(4\pi)^2 R^4} 
\end{equation}
By using \autoref{eq:effectivearea} we can write the received power as
\begin{equation}
	P_r = \frac{\powtransmitter G_t G_r \sigma \lambda^2 }{(4\pi)^3 R^4} 
\end{equation}
Therefore the received power depends on both the gain of the transmitter and receiver,
the property of the target, encoded in the radar cross section, its distance,
and the wave length.


\section{EM waves}\label{sec:emwaves}

Electromagnetic waves consist of oscillating electric and magnetic fields,
which are perpendicular to each other and propagate through space in the
direction perpendicular to the plane formed by the two fields. The direction of
the electrical field is said to be the \emph{polarization} of the wave.
Maxwell’s equations describe the behavior of these fields
\cite[p.~5]{richards2010principles}. 

An electromagnetic wave is described by its \emph{wavelength} $\lambda$ and its
\emph{frequency}. The two quantities are related by:
\begin{equation}\label{eq:wavelength}
	\lambda = \frac{v}{f}
\end{equation}
where $v$ is the speed of light in the medium. The speed of light in vacuum is 
denoted $c$ and is $299792458~\text{m}/\text{s}$.

Two electromagnetic waves having the same frequency that are in a given moment
in the same place interfere with each other, resulting in a new wave with
either a greater amplitude (constructive interference), or smaller amplitude
(destructive interference).

\section{Scattering Physics}

In this section we describe how an EM wave interact with a body. The interaction
behaves differently based on the size $L$ of the object relative to the wave length
$\lambda$ of the incident radiation. Three different scattering regimes are
recognized \cite{richards2010principles}:
\begin{itemize}
	\item when $L \ll \lambda$, Rayleigh scattering;
	\item when $L \approx \lambda$, optics scattering with creeping surface waves;
	\item when $L \gg \lambda$, optics scattering.
\end{itemize}

While radar systems in principle can work at any frequencies, more commonly
microwave bands between 1 and 40 GHz are used \cite{radayanalysisandmodeling}.
This means that the wave length $\lambda$ varies from 0.75 cm to 30 cm,
justifying assuming the optics scattering regime for radar applications.
We therefore present the relevant optical phenomenons.

\subsection{Electromagnetic Properties of Materials}

The propagation of EM waves depends on the properties of the medium they
propagate in. Three frequency-dependant parameters are used to characterize a
material \cite[p.~27]{electromagnetics-vol1}: 
\begin{itemize}
	\item permittivity, which quantifies how the material affect the intensity of the
		electric field in response to charges. A greater permittivity means that, 
		the same charges result in a weaker electric
		field\cite[p.~20]{electromagnetics-vol1};
	\item permeability, which relates the magnetic field to current;
	\item conductivity, which determines the current density in response to an
		electric field. 
\end{itemize}

\subsection{Reflection and Transmission}

When light falls on the boundary between two homogeneous media with different
properties, it is split in two waves: a reflected one, remaining in the first
medium, and a transmitted one, also called refracted, entering the second
medium. The direction of the transmitted and reflected waves obey
Snell's Law \cite{itu-buildings}:
\begin{equation}\label{snell}
	\frac{\sin(\theta_i)}{v_1} = \frac{\sin(\theta_r)}{v_1} = \frac{\sin(\theta_t)}{v_2}
\end{equation}
where $\theta_i$ is the angle formed by the incident light and the normal of
the surface, $\theta_r$ is the angle formed by the reflected light and the normal,
$\theta_t$ is the angle formed by the transmitted light and the negated normal, 
$v_1$ is the phase velocity of the light in the first medium and $v_2$ is the
phase velocity in the second one. The incident, reflected, transmitted and normal
directions all lay in the same plane. \autoref{snell} can also be
expressed with the indices of refraction. The index of refraction $n$ for a
medium in which the light has phase velocity $v$ is:
\begin{equation}\label{iorspeed}
	n = \frac{c}{v}
\end{equation}
where $c$ is the speed of light in vacuum. Therefore from
\autoref{snell}, solving for $\sin(\theta_t)$ we get:
\begin{equation}\label{snell2}
	\sin(\theta_t) = \frac{v_2}{v_1}\sin(\theta_i) = \frac{n_1}{n_2}\sin(\theta_i)
\end{equation}


Since the angles with the normal are
between 0 and $\frac{\pi}{2}$, \autoref{snell} implies $\theta_i
= \theta_r$. \autoref{fig:refraction} shows an example of reflection and
transmission as predicted by Snell's law.

\begin{figure}
	\centering
	\begin{tikzpicture}
		\newcommand*{\incomingangle}{60}
		\newcommand*{\len}{5cm}
		\tikzset{
			myarrow/.tip={Triangle[length=2.5mm, width=2mm]},
		}
		\coordinate (n1) at (0, \len);
		\coordinate (n2) at (0,-\len);
		\coordinate (r1) at (canvas polar cs:angle=\incomingangle+90,radius=\len);
		\coordinate (r2) at (canvas polar cs:angle=90-\incomingangle,radius=\len);
		\coordinate (r3) at (canvas polar cs:angle={-90 + asin(sin(\incomingangle / 2.00))},radius=\len);
		\coordinate (O) at (0,0);
		\fill[fill=gray!20] (-\len, 0) rectangle (\len,-\len);
		\pic[draw, fill=green!30, angle eccentricity=1.5, "$\theta_i$"] {angle = n1--O--r1};
		\pic[draw, fill=green!30, angle eccentricity=1.5, "$\theta_r$"] {angle = r2--O--n1};
		\pic[draw, fill=green!30, angle eccentricity=1.5, "$\theta_t$"] {angle = n2--O--r3};
		\draw (-\len,0) -- (\len,0);
		\draw[dashed] (n2) -- (n1);
		\draw[-myarrow] (r1) node [below] {$\vec d_i$} -- (O);
		\draw[-myarrow] (O)  -- (r2) node [below] {$\vec d_r$};
		\draw[-myarrow] (O) -- (r3) node [above right] {$\vec d_t$} ;
		\draw (4,1) node {Medium 1}; 
		\draw (4,-1) node {Medium 2}; 

		\undef\incomingangle
		\undef\len
	\end{tikzpicture}
	\caption{Direction of incoming $\vec d_i$, reflected $\vec d_r$ and
	transmitted $\vec d_t$ waves at a boundary between two media according to
	Snell's law. The dotted line is the direction normal to the
	surface. In particular, in the pictured scenario the second medium has an
	index of refraction bigger than the first medium. Therefore we can note
	that the transmitted wave is deflected closer to the normal of the
	surface.}
	\label{fig:refraction}
\end{figure}

We are also interested in how the power of the incident light is distributed
among the reflected and transmitted waves. The power of the reflected light 
for an isotropic medium with conductivity $\sigma = 0$ and permeability equals
to the permeability of free space is related to the incident power by the
Fresnel equations \cite{itu-buildings}:
\begin{gather}\label{eq:fresnel}
	\frac{P_{t\parallel}}{P_{i\parallel}} = \left|\frac{n_2\cos(\theta_i) - n_1 \cos(\theta_t)}{n_2\cos(\theta_i) + n_1 \cos(\theta_t)}\right|^2 \\
	\frac{P_{t\perp}}{P_{i\perp}} = \left|\frac{n_1\cos(\theta_i) - n_2 \cos(\theta_t)}{n_1\cos(\theta_i) + n_2 \cos(\theta_t)}\right|^2
\end{gather}
where $P_{t\parallel}$ ($P_{i\parallel}$) is the power of the transmitted
(incident) wave with polarization parallel to the incident plane and
$P_{t\perp}$ ($P_{i\perp}$) is the power of the transmitted (incident) wave
with polarization perpendicular to the incident plane, $n_1$ is the index of
refraction of the first medium and $n_2$ is the index of refraction of the second 
medium. The index of refraction, apart from the formulation given in
\autoref{iorspeed}, can also be expressed as a function of the relative
permittivity $\epsilon_r$ and relative permeability $\mu_r$ of the medium
\cite{itu-buildings}:
\begin{equation}\label{eq:real-ior}
	n = \sqrt{\epsilon_r}
\end{equation}
Furthermore, for a medium with non zero conductivity $\sigma$, 
we can introduce a complex index of refraction defined as \cite{itu-buildings}:
\begin{equation}\label{eq:complex-ior}
	n = \sqrt{\epsilon_r - j \frac{\sigma}{\epsilon_0 \omega}}
\end{equation}
where $j$ is the imaginary unit, $\sigma$ is conductivity of the material,
$\epsilon_0$ is the permittivity of free space and $\omega$ is the angular
frequency defined as $\omega = 2\pi f$ with $f$ the frequency. It can be noted
that for $\sigma = 0$ the index of refraction reduces to \autoref{eq:real-ior}.

\subsubsection{Total reflection}

\autoref{snell2} does not have a solution for the transmission
angle $\theta_t$ when the index of refraction of the second medium is smaller
than the index of the first and for a large enough incidence angle $\theta_i$.
When those conditions are verified, all the light is reflected and no light is
transmitted. This phenomenon is called total reflection \cite[p.~50]{born}.

The limiting angle of incidence $\overline{\theta}_i$, which when exceeded total
reflection arises, is given by
\begin{gather}
	% \frac{n_1}{n_2}\sin(\overline{\theta}_i) = 1 \\
	\overline{\theta}_i = \arcsin\left(\frac{n_2}{n_1}\right)
\end{gather}

%TODO immagine

\subsubsection{Dependency of material properties on the EM wave frequency}\label{subsub:prop-fit}

\autoref{eq:fresnel} gives a way to compute the reflected and transmitted power
given the angle of incidence and index of refraction. In turns, the index of
refraction depends on the permittivity and conductivity of the material
(\autoref{eq:complex-ior}). However, the same material can behave differently
for different frequencies, thus, in general, permittivity and conductivity
depend on the frequency. Therefore, to derive those electrical properties for a
given frequency, models based on the fitting of real world measures have been
developed. The International Telecommunication Union proposes 
a model based on four parameters $a$, $b$, $c$ and $d$ from which 
permittivity and conductivity are obtained \cite{itu-buildings}:
\begin{gather}
	\epsilon_r = a f_{\text{GHz}} ^ b \\
	\sigma = c f_{\text{GHz}} ^ d 
\end{gather}
where $f_{\text{GHz}}$ is the frequency in giga Hertz. Example of parameters
computed to fit real world measures are shown in
\autoref{table:material-props}, while \autoref{fig:reflectivity} pictures how
the reflectivity is affected by the frequency of the incoming EM wave for the
concrete and wet ground materials.

\begin{table}
	\centering\begin{tabular}{c|cccc|c}
		\toprule
		\textbf{Material class} & $a$ & $b$ & $c$ & $d$ & \textbf{Frequency range (GHz)} \\
		\hline
		Concrete & 5.24 & 0 & 0.0462 & 0.7822 & 1-100 \\
		Wood & 1.99 & 0 & 0.0047 & 1.0718 & 0.001-100 \\
		Metal & 1 & 0 & $10^7$ & 0 & 1-100 \\
		Very dry ground & 3 & 0 & 0.00015 & 2.52 & 1-10 \\
		Medium dry ground & 15 & -0.1 & 0.035 & 1.63 & 1-10 \\
		Wet ground & 30 & -0.4 & 0.15 & 1.30 & 1-10 \\
		\bottomrule
	\end{tabular}
	\caption{Parameters used to compute permittivity and conductivity for
	different materials as recommended by the International Telecommunication
	Union \cite{itu-buildings}. The frequency range corresponds to the
	frequencies used in the measures the parameters are based on.}
	\label{table:material-props}
\end{table}

\begin{figure}
	\centering
	\begin{subfigure}[t]{0.45\textwidth}
		\includegraphics[width=\textwidth]{images/concrete.pdf}
		\caption{Concrete material. The reflectivity does not appear to change
		significantly for 1 and 10 GHz.}
	\end{subfigure}
	\hfill
	\begin{subfigure}[t]{0.45\textwidth}
		\includegraphics[width=\textwidth]{images/wet_ground.pdf}
		\caption{Wet ground material. The reflectivity changes significantly
		for 1 and 10 GHz, especially at low angles of incidence. }
	\end{subfigure}

	% \hfill

	\caption{Reflectivity, given by $\frac{1}{2}
	\left(\frac{P_{t\parallel}}{P_{i\parallel}} +
	\frac{P_{t\perp}}{P_{i\perp}}\right)$, as a function of angle of incidence
	for different frequencies and different materials with the parameters given in \autoref{table:material-props}.}
	\label{fig:reflectivity}
\end{figure}



\subsubsection{Diffusion}

When a surface is rough compared to the wavelength, the reflection is specular
only in a small region. This results, at macroscopic scale, in the wave being
reflected at all angles. This phenomenon is called diffusion or diffuse
scattering \cite[p.~17]{richards2010principles}.

\subsection{Diffraction}

EM waves travel in a straight line in an homogeneous medium and are reflected
or refracted when meeting a boundary between two different medium. However, EM
waves can also interact with object when traveling close to them, resulting in
a bending of the direction of travel towards the geometric shadow region. This
phenomenon is called \emph{diffraction} \cite[p.~140]{richards2010principles}.



\subsection{Propagation of EM waves in the atmosphere}

The atmosphere causes an attenuation of the EM waves that propagate into it.
Two major factor in the attenuation are absorption and scattering
\cite[p.~121]{richards2010principles}. Absorption is caused by lossy media in
the atmosphere, such as oxygen molecules and raindrops, and converts part of
the energy of the wave into heat. Scattering in the atmosphere causes the EM
wave to be reflected in all directions. The atmospheric attenuation is modeled
to depends on an attenuation coefficient $\alpha$ measured in $\text{m}^{-1}$.
Thus, the attenuation, expressed as the ratio of the attenuated power $P_a$ and the
original power $P_0$, is given by:
\begin{equation}
	\frac{P_a}{P_0} = 10^{\alpha \frac{L}{2}}
\end{equation}
where $L$ is the path length in meters. The attenuation coefficient depends on
atmospheric conditions, such as rain intensity, the frequency of the
EM wave but also temperature and altitude.


\section{GPU Programming and the Vulkan API}\label{sec:vulkan}

Vulkan is a graphics and compute API, designed to provide cross-platform access
to modern GPUs \cite{whatisvulkan}. The Vulkan API is defined by a public
specification \cite{vulkanspec} written by the Khronos Group, a non profit
organization developing royalty-free standards \cite{aboutkhronos}.

In Vulkan, shaders are programs run on the GPU. They can be written in high
level languages such as GLSL or HLSL and compiled to a platform-independent
bytecode format, called SPIR-V. Shaders are used in the context of a
\emph{pipeline}. A pipeline constitutes a configuration for how the shaders are
to be used and can be one of three different kinds:
\begin{itemize}
	\item graphics pipeline: used for 2D or 3D rendering;
	\item compute pipeline: used for general purpose GPU programming (GPGPU);
	\item ray tracing pipeline: used to perform ray tracing. 
\end{itemize}
In this work we are mainly interested in the ray tracing pipeline.

\subsection{The ray tracing pipeline}\label{subsec:raytracing}

The Vulkan ray tracing pipeline, introduced by the extension
\verb!VK_KHR_ray_tracing_pipeline! \cite{vulkanraytracingpipeline}, leverages a
dedicated set of shader stages and commands, independent from both the graphics
and compute pipelines. \autoref{fig:raytracingpipeline}
pictures
the five stages supported by the ray tracing pipeline and the interaction among
them.

\tikzstyle{programmable} = [rectangle, rounded corners, minimum width=3cm, minimum height=1cm,text centered, draw=black, fill=blue!30]
\tikzstyle{fixed} = [rectangle, minimum width=3cm, minimum height=1cm, text centered, draw=black, fill=lightgray!30]
\tikzstyle{arrow} = [->,>=stealth]
\tikzstyle{decision} = [diamond, minimum width=1cm, minimum height=1cm, text centered, draw=black, fill=white]
\begin{figure}\centering
	\begin{tikzpicture}[node distance=2cm]
		\node (raygen) [programmable] {Ray Generation shader};
		\node (astraversal) [fixed, below=of raygen] {Acceleration Structure Traversal};
		\node (is) [programmable, right=of astraversal] {Intersection shader};
		\node (any) [programmable, above=of is, yshift=-1cm] {Any Hit shader};
		\node (maybehit) [decision, below=of astraversal, yshift=1cm] {Hit?};
		\node (miss) [programmable, right=of maybehit] {Miss shader};
		\node (closest) [programmable, left=of maybehit] {Closest Hit shader};
		\draw [arrow] (raygen) -- (astraversal);
		\draw [arrow] (astraversal) -- (maybehit);
		\draw [arrow] (maybehit) -- (closest) node [midway, above] {Yes};
		\draw [arrow] (maybehit) -- (miss) node [midway, above] {No};
		\draw [arrow] (astraversal) -- (is);
		\draw [arrow] (is) -- (any);
		\draw [arrow] (any) -| ($(astraversal.north) + (2cm,0)$);
	\end{tikzpicture}
	\caption{Ray tracing pipeline stages. Stages in blue are fully programmable
	with a shader, while the stage in gray is fixed.}
	\label{fig:raytracingpipeline}
\end{figure}

% Explain the various stages

\subsubsection{Ray generation shader stage}\label{subsubsec:raygen}

The ray generation shader is the starting point of the pipeline. A three
dimensional grid of shader invocations is launched with the command
\verb!vkCmdTraceRaysKHR!. A ray is an half-line expressed by the parametric
equation:
\begin{equation}\label{eq:ray}
	r(t) = \vec{o} + t \vec{d}
\end{equation}
where $\vec{o}$ is the origin of the ray, $\vec{d}$ is its direction and $t \ge
0$.

In the shader, each invocation can access to its launch identifier with the
built-in variable \verb!gl_LaunchIDEX! and trace a ray with the
\verb!traceRayEXT! function, passing its origin and direction as defined in
\autoref{eq:ray}, together with a minimum ($t_{min}$) and
maximum ($t_{max})$ value of the parameter, defining an interval in which to
search for intersection with the geometries.

Each ray is associated with a
\emph{payload}, an arbitrary, application-defined data structure which can be
read and written from the various stages of the pipeline.


\subsubsection{Acceleration structure traversal}

An acceleration structure is an implementation-dependant and opaque data
structure used to represent geometric objects and efficiently find which, if
any, primitive is intersected by the traced ray. An acceleration structure can 
either be a bottom level acceleration structure (BLAS), in which case it 
can either store an array of geometries consisting of triangles or axis aligned
bounding boxes, or can be a top level acceleration structure (TLAS), which 
contains an array of BLAS.
During the traversal, an intersection with triangles for parameters in the range
$(t_{min}, t_{max})$ causes invocation of the any hit shader, while an
intersection with an axis aligned bounding box (AABB) causes first invocation
of the intersection shader and possibly then of the any hit shader.


\subsubsection{Intersection shader stage}\label{subsub:intersectionshader}

Intersection shader are used to implement intersections with user-defined
primitives, such as surfaces defined by an equation instead of triangles. When
an intersection with the corresponding AABB is found, this shader is called and
tests whether the primitive is intersected by the ray. If the intersection has
occurred, it uses the \verb!reportIntersectionEXT! function and the
intersection is reported to the any hit shader. Intersection shaders cannot
read or modify the ray payload.


\subsubsection{Any hit shader stage}

An any hit shader is an optional shader whose goal is to confirm or ignore a
reported intersection with an object. This stage can also modify the ray
payload or terminate the tracing of the ray early. A possible use for this
shader is to have partially transparent surface, where the transparency is
defined by a texture.


\subsubsection{Closest hit shader stage}\label{subsubsec:closesthitshader}

When all the intersection of a ray has been determined, the closest hit shader is
called for the closest one (i.e.~the one with minimum value of the parameter
$t$), if it exists. In this stage the payload can be modified and it is also
possible to recursively trace new rays with \verb!traceRayEXT!.
Shaders in this stage have access to useful built-in variables:
\begin{itemize}
	\item \verb!gl_InstanceID!, containing the index of the BLAS instance the
		intersection happened with;
	\item \verb!gl_GeometryIndexID!, containing the index of the geometry in the
		BLAS;
	\item \verb!gl_PrimitiveID!, the index of the intersected triangle.
\end{itemize}
Together, those three variable allows to unambiguously identify with which triangle
the intersection happened.


\subsubsection{Miss shader stage}

When no intersection is detected, the miss shader. In this shader the ray
payload can be written to.


\subsubsection{Shader binding table}\label{subsec:sbt}

A ray tracing pipeline may contain more than one shader of the same kind, for
example multiples closest hit shaders. In that case, the acceleration structure
traversal needs additional information in order to choose which shader to call.
This information is available in the form of the \emph{shader binding table}
\cite[p.~192]{gemsII}.
The shader binding table stores handles which refer to the shader records, possibly
together with embedded parameters. There are three
different kinds of shader records: the ray generation record, which contains
just the ray generation shader, the miss record, containing the miss shader,
and the hit group record, which can contain the closest hit, any hit and
intersection shaders. The mechanism used for the selection of a shader record
depends on the shader record type and it is explained in details in the
following paragraphs.

For the hit group record, the selected group is given by the hit group record
index $H_\text{index}$:
\begin{equation}
	H_\text{index} = I_\text{offset} + R_\text{offset} +
	R_\text{stride} \cdot G_\text{id}
\end{equation}
where $I_\text{offset}$ is specified for each BLAS when building a TLAS with
the \texttt{instanceShaderBindingTableRecordOffset} field in the structure
\texttt{VkAccelerationStructureInstanceKHR}, $R_\text{offset}$ and $R_\text{stride}$
are both passed as argument to the function \texttt{traceRayEXT}, $G_\text{id}$
is the index of the geometry in BLAS. This allows to control the used shader group
both on a per-object basis, with $I_\text{offset}$ and $G_\text{id}$, and on a
per-ray basis, with $R_\text{offset}$ and $R_\text{stride}$.


The miss group shader to use is simply specified by the argument
\texttt{missIndex} in \texttt{traceRayEXT}.

\subsection{Comparison between Vulkan and other GPU programming API and platform}

OpenGL is an other graphic API with wide support across hardware vendors and
operating systems, whose standard is maintained by the Khronos Group.
Furthermore, since version 4.3, it supports GPGPU with compute shaders
\cite{opengl43}. OpenGL, however, does not support hardware-accelerated ray
tracing \cite{unterguggenberger2023vulkan}, rendering it unsuitable for this
work.

DirectX is a collection of API designed by Microsoft and available on the
Microsoft platforms. Since version 12, DirectX includes support for hardware
accelerated ray tracing \cite{gemsdx}. Using DirectX, however, limits
application support to the Microsoft platforms such as the Windows operating
system.

OptiX \cite{optix} is a proprietary framework for ray tracing developed by
NVIDIA. It supports Windows, Linux and OSX but only NVIDIA GPUs of the Maxwell
generation or newer.

\autoref{gpuapi} 
summarize the differences relevant for this work of the discussed GPU APIs.

\begin{table}
	\begin{center}
		\begin{tabular}{|c|c|c|c|c|}
			\hline
			& \thead{OS agnostic} & \thead{Hardware vendor \\ agnostic} & \thead{GPGPU programming \\ support} & \thead{Support for hardware \\ accelerated ray-tracing} \\
			\hline\hline
			% CUDA & \textbullet & & \textbullet & \\
			OpenGL & \textbullet & \textbullet & \textbullet & \\
			\hline
			DirectX 12 & & \textbullet & \textbullet & \textbullet \\
			\hline
			OptiX & \textbullet & & \textbullet & \textbullet \\
			\hline
			Vulkan & \textbullet & \textbullet & \textbullet & \textbullet \\
			\hline
		\end{tabular}
	\end{center}
	\caption{Summary of GPU programming APIs support for different operating
	systems, hardware vendors and hardware ray-tracing acceleration.}
	\label{gpuapi}
\end{table}


\section{Hardware accelerated ray tracing}\label{sec:hw-rt}

In September 2018, NVIDIA introduced in theirs GPU, with the Turing
architecture, the \emph{RT Cores}. RT Cores are dedicated to the acceleration
of the traversal of bounding volume hierarchies (BVH) and the testing of
ray/triangle intersection \cite{turing-whitepaper}. RT Cores therefore allow to
offload the Streaming Multiprocessor, which meanwhile can carry out other
computations. Sanzharov et al.~\cite{rtxexamination} show that ray tracing
implementation using the new RT Cores are from 2 to 5 times faster with respect
to existing software based ray tracers. They observe that the higher speedups
happen with rays with random directions, which require random memory access to
traverse the BVH. Therefore they speculate that the new hardware is capable of
reordering memory accesses caused by BVH traversal.

AMD as well introduced in their RDNA2 architecture ray accelerators.
RDNA2 ships with one ray accelerator with each of their Compute Units (CUs) \cite{rdna2}.
Those ray accelerators execute two new instructions, dedicated to the intersection test
of bounding volume hierarchies and triangles with rays \cite{rdna2}.
